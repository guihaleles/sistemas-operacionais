Este relatório descreve a prática de controle da planta de um rotor duplo. O sistema é composto por duas hélices, semelhante a um helicóptero, sendo uma no sentido horizontal e outra no vertical. O intuito do presente trabalho foi realizar o controle de ambos graus de liberdade da planta: ângulos de arfagem e de guinada. Para tal, realizou-se identificação do sistema para cada entrada e para cada saída. Assim foram obtidas quatro funções de transferência que aproxima o sistema real. Tal modelo foi validado por meio de simulações na planta. Foi realizado o projeto de um desacoplador e controladores PID foram projetados para controle de arfagem e guinada. Esses controladores foram simulados nos modelos e posteriormente testado na planta.